\documentclass[12pt]{article}
\usepackage{a4,amsmath,latexsym,amssymb,textcomp,graphicx,fancyhdr,fancybox,listings,times,paralist,tikz,pgfplots,longtable,color,wrapfig}
\usepackage[ngerman]{babel}
\usepackage[utf8]{inputenc}
\usepackage[T1]{fontenc}
\usepackage[top=2cm,right=2cm,bottom=2cm,left=2cm]{geometry}
\usepackage[colorlinks=false,urlcolor=blue]{hyperref}

\definecolor{hsred}{HTML}{D9230F}
\definecolor{white}{HTML}{FFFFFF}

% Header and footer styling
\renewcommand{\headrulewidth}{0pt}
\renewcommand{\footrulewidth}{0pt}
\pagestyle{fancy}
\fancyhf{}
\fancyhead[C]{
	\begin{tikzpicture}[overlay,remember picture]
		\fill[hsred] (current page.north west) rectangle ($(current page.north east)+(0,-10mm)$);
		\node[anchor=north west, text=white, minimum size=5mm, inner xsep=18.5mm, inner ysep=3.5mm] at (current page.north west) {HoneySens Administrationshandbuch};
		\node[anchor=north east, text=white, minimum size=5mm, inner xsep=18.5mm, inner ysep=3.5mm] at (current page.north east) {\thepage};
	\end{tikzpicture}
}
\fancyfoot[C]{}

\newcommand{\bigcell}[2][c]{\begin{tabular}[#1]{@{}l@{}}#2\end{tabular}}
\usetikzlibrary{calc}

% ToC styling - display sections as subsections and hide subsections
\makeatletter
\renewcommand*\l@section{\@dottedtocline{1}{1.5em}{2.3em}}
\makeatother
\setcounter{tocdepth}{1}

\begin{document}
\setlength\parindent{0pt}
\setlength\parskip{1em}
\renewenvironment{itemize}{\begin{compactitem}}{\end{compactitem}}
\renewenvironment{enumerate}{\begin{compactenum}}{\end{compactenum}}
\pgfplotsset{/pgf/number format/use comma}
\pgfplotsset{compat=1.5}
\definecolor{mygreen}{rgb}{0,0.6,0}
\lstset{basicstyle=\ttfamily,frame=single,commentstyle=\color{darkgray}\bfseries}
\lstdefinelanguage{JavaScript}{
  keywords={typeof, new, true, false, catch, function, return, null, catch, switch, var, if, in, while, do, else, case, break},
  keywordstyle=\color{darkgray}\bfseries,
  ndkeywords={class, export, boolean, throw, implements, import, this},
  ndkeywordstyle=\color{darkgray}\bfseries,
  identifierstyle=\color{darkgray},
  sensitive=false,
  comment=[l]{//},
  morecomment=[s]{/*}{*/},
  commentstyle=\color{purple}\ttfamily,
  stringstyle=\color{black}\ttfamily,
  morestring=[b]',
  morestring=[b]"`
}
\begin{titlepage}
	\begin{tikzpicture}[overlay,remember picture]
		\fill[hsred] (current page.north west) rectangle ($(current page.north east)+(0,-2mm)$);
	\end{tikzpicture}
	\begin{tikzpicture}[overlay,remember picture]
		\fill[hsred] ($(current page.north west)+(0,-50mm)$) rectangle ($(current page.north east)+(0,-130mm)$);
	\end{tikzpicture}
	\par
	\vspace{4.4cm}
	\begin{wrapfigure}{r}{0.25\textwidth}
	\vspace{-2.0cm}
		\centering
		\includegraphics[width=0.25\textwidth]{./graphics/honeysens-logo-white.png}\par
	\end{wrapfigure}
	{\color{white}
	{\fontsize{1.3cm}{1.5cm}\selectfont Administration}\par
		\vspace{0.5cm}
		{\large für Version 0.2.1}\par
		{\large Stand: November 2016}
	}
	\vfill
	geschrieben von\\
	Pascal Brückner
\end{titlepage}

\pagenumbering{Roman}
\tableofcontents

\newpage
\pagenumbering{arabic}

\section{Installation/Update Serverkomponente}
Die Serversoftware wird in Form eines \textit{Docker}-Containers ausgeliefert. Für die Nutzung von Docker bitte die Dokumentation auf \url{https://docker.com/} berücksichtigen.

\textbf{Hinweis:} In den nachfolgenden Listings bezeichnet \verb|x.y.z| die jeweils aktuelle Version der Serversoftware und muss entsprechend angepasst werden!

Registrieren der (aktualisierten) Server-Software:\\
\begin{figure}[h]
	\begin{lstlisting}
$ docker load -i ./HoneySens-Server-x.y.z.tar.gz
	\end{lstlisting}
\end{figure}

Falls es sich um eine Neuinstallation des Servers handelt, muss zunächst ein sog. \textit{Data-only Container} angelegt werden, in dem alle Serverdaten abgelegt werden (u.a. die Inhalte der Datenbank, Sensor-Firmware, die Serverkonfiguration usw).\\
\begin{figure}[h]
	\begin{lstlisting}
$ docker run --name honeysens-data 
             --entrypoint /bin/echo pbrueckner/honeysens:x.y.z 
             Data-only container for honeysens
	\end{lstlisting}
\end{figure}

Falls hingegen ein Update einer bereits laufenden Server-Software durchgeführt werden soll, ist es zunächst notwendig, den bestehenden Container mit dem Namen \textit{honeysens} zu entfernen:
\begin{figure}[h]
	\begin{lstlisting}
$ docker stop honeysens
$ docker rm -v honeysens
	\end{lstlisting}
\end{figure}

Anschließend wird ein neuer Container mit der aktuellen Server-Software gestartet. Die Pfade für das Serverzertifikat \verb|cert.pem| und den zugehörigen privaten Schlüssel \verb|key.pem| müssen auf die entsprechenden Dateien im lokalen Serversystem (außerhalb des Containers) verweisen. Es ist hierbei wichtig, dass das Zertifikat die \textbf{gesamte Zertifikatskette} vom Serverzertifikat bis zum Root-Zertifikat beinhaltet, damit die Sensoren später den Server korrekt authentifizieren können. Wenn die beiden \verb|--volume|-Optionen weggelassen werden, erzeugt der Server automatisch ein selbstsigniertes Zertifikat.
\begin{figure}[!h]
	\begin{lstlisting}
$ docker run -d -p 80:80 -p 443:443 --name honeysens
  --volumes-from honeysens-data
  --privileged
  --volume /path/to/cert.pem:/opt/HoneySens/data/ssl-cert.pem
  --volume /path/to/key.pem:/opt/HoneySens/data/ssl-cert.key
  pbrueckner/honeysens:x.y.z
	\end{lstlisting}
\end{figure}

Der Server muss über TCP-Port 80 (HTTP) und 443 (HTTPS) für die Sensoren erreichbar sein. HTTP(S)-Proxy-Server werden unterstützt.


\end{document}
